%%
\documentclass[11pt]{article}
\usepackage[english]{babel}
\usepackage[a4paper]{geometry}
\usepackage{amssymb,amsmath,amsthm}
\usepackage{graphicx,wrapfig}
\usepackage{latexsym}
\usepackage{mathtools} 
\usepackage{hyperref}
\usepackage{fontspec}
\setmainfont{Cochin}
\usepackage[italic]{mathastext}
%
\usepackage{cancel}
%
\usepackage[backend=bibtex,sorting=none,giveninits=true]{biblatex}
\bibliography{../../../../Latex/Bibliotheque/bibliotheque.bib}
%
%---------------------------------------------------------
%\newcommand{\Set}[1]{\left\{#1\right\}} 
%\newcommand{\transpose}[1]{#1^{\mathsf{T}}} 
%\renewcommand{\div}{\operatorname{div}} 
%\newcommand{\dn}[1]{\frac{\partial #1}{\partial n}}
%\newcommand{\GammaD}{\Gamma_{\rm D}}
%\newcommand{\GammaN}{\Gamma_{\rm N}}
%\newcommand{\udir}{u^{\rm D}}
\newcommand{\qd}{q^{\rm D}}
\input{../../../../Latex/Articles/macros.tex}
%---------------------------------------------------------
%
%---------------------------------------------------------
%
%
%==========================================
\title{SimFem}
\author{Roland Becker}
%
%==========================================
\begin{document}
%==========================================
\maketitle
\setcounter{tocdepth}{3}
\tableofcontents
%
%
%==========================================
\section{Geometry and finite elements}\label{sec:}
%==========================================
%
%
%-------------------------------------------------------------------------
\subsection{Simplices}\label{subsec:}
%-------------------------------------------------------------------------
%
We consider an arbitrary non-degenerate simplex $K=(x_0,x_1,\ldots, x_{d})$. The (signed) volume of $K$ is given by
%
\begin{equation}\label{eq:}
|K| = \frac{1}{d!} \det(x_1-x_{0},\ldots, x_{d}-x_{0})= \frac{1}{d!} \det(1,x_{0},x_1\ldots, x_{d})\quad 1=\transpose{(1,\ldots,1)}.
\end{equation}
%
The $d+1$ sides $S_k$ (co-dimension one, $d-1$-simplices or facets) are defined by
$S_k=(x_0,\ldots, \cancel{x_k}, \ldots, x_{d})$. The height is $d_k=|P_{S_k}x_k - x_k|$, where $P_S$ is the orthogonal projection on the hyperplane associated to $S$. We have
%
\begin{align*}
d_k = d\frac{|K|}{|S_k|} \qquad\mbox{(and for $d=3 \; |S_k| = \frac12 |u\times v| $)}
\end{align*}
%
%
%-------------------------------------------------------------------------
\subsection{Finite elements}\label{subsec:}
%-------------------------------------------------------------------------
%
%
The $d+1$ basis functions of the Courant element are the barycentric coordinates 
$\lambda_i$ defined as being affine with respect to the coordinates and $\lambda_i(x_j)=\delta_{ij}$. The constant gradient is given by
%
\begin{align*}
\nabla \lambda_i = - \frac{1}{d_i}\vec{n_i}. 
\end{align*}
%
The relation with the $d+1$ Crouzeix-Raviart basis functions $\psi_i$ is given by 
%
\begin{align*}
\psi_i = 1 - d\lambda_i
\end{align*}
%
Finally the $d+1$ Raviart-Thomas basis functions $\xi_i$, associated to side $S_i$, i.e. the opposite node $x_i$, are given by 
%
\begin{align*}
\xi_i = \frac{x-x_i}{d_i} = \frac{1}{d_i}\sum_{j=0\atop i\ne j}^d x_j \lambda_j
\end{align*}
%

%
%-------------------------------------------------------------------------
\subsection{Numerical integration}\label{subsec:}
%-------------------------------------------------------------------------
%
Any polynomial in the barycentric coordinates can be integrates exactly.
%
\begin{equation}\label{eq:}
\int_K \prod_{i=1}^{d+1}\lambda_i^{n_i} \,dv = d!|K|\frac{\prod\limits_{i=1}^{d+1} n_i!}{\left( \sum\limits_{i=1}^{d+1} n_i + d\right)!}
\end{equation}
%
see \cite{EisenbergMalvern73}, \cite{VermolenSegal18}.
%

Let $V=\vect{\phi}$. For a smooth function $f$ and $u=\sum_j u_j \phi_j$ we use and approximation based on $\vect{\psi}$ such that $\psi_l(x_k-=\delta_{kl}$ and
%
\begin{align*}
f(u) \approx \sum_k f(u(x_k))\psi_k = \sum_k f(\sum_j u_j \phi_j(x_k))\psi_k
\end{align*}
%
Then
%
\begin{align*}
\int_K f(u) \phi_i \approx \sum_k f_k\int_K \psi_k\phi_i,\quad f_k = f(\sum_l u_l \phi_l(x_k))\\ 
\int_K f'(u)(\phi_j) \phi_i \approx \sum_k f'_{k,j}\int_K  \psi_k\phi_i,\quad f'_{k,j} = f'(\sum_l u_l \phi_l(x_k))\phi_j(x_k)
\end{align*}
%
For $\psi=\phi$ this becomes considerably cheaper:
%
\begin{align*}
\int_K f(u) \phi_i \approx \sum_k f_k\int_K \phi_k\phi_i,\quad f_k = f(u_k)\\ 
\int_K f'(u)(\phi_j) \phi_i \approx f'_j\int_K  \phi_j \phi_i,\quad f'_k = f'(u_k)
\end{align*}
%


%
%-------------------------------------------------------------------------
\subsection{Element matrices for $C^1$}\label{subsec:}
%-------------------------------------------------------------------------
%
%
%~~~~~~~~~~~~~~~~~~~~~~~~~~~~~
\subsubsection{Mass matrix}
%~~~~~~~~~~~~~~~~~~~~~~~~~~~~~
%
%
\begin{align*}
M_{ij} = \begin{cases}
\frac{2d!|K|}{(2+d)!}& $i=j$\\
\frac{d!|K|}{(2+d)!}& $i\ne j$
\end{cases}\quad
M^{\rm 1D} =
|K| 
\begin{bmatrix}
\frac13 & \frac16\\
\frac16 & \frac13
\end{bmatrix}, \; 
M^{\rm 2D} =
|K| 
\begin{bmatrix}
\frac16 & \frac{1}{12} & \frac{1}{12}\\
\frac{1}{12} & \frac16 & \frac{1}{12}\\
\frac{1}{12} & \frac{1}{12} & \frac16
\end{bmatrix}, \; 
M^{\rm 3D} =
|K| 
\begin{bmatrix}
\frac{1}{10} & \frac{1}{20} & \frac{1}{20} & \frac{1}{20}\\
\frac{1}{20} & \frac{1}{10} & \frac{1}{20} & \frac{1}{20}\\
\frac{1}{20} & \frac{1}{20} & \frac{1}{10} & \frac{1}{20}\\
\frac{1}{20} & \frac{1}{20} & \frac{1}{20} & \frac{1}{10}
\end{bmatrix}
\end{align*}
%

Lumped mass
\begin{align*}
\tilde M_{ij} = \begin{cases}
\frac{|K|}{(1+d)} & $i=j$\\
0& $i\ne j$
\end{cases}
\end{align*}

%
%==========================================
\section{Test problems}\label{sec:}
%==========================================
%
%
%-------------------------------------------------------------------------
\subsection{Advection-Diffusion-Reaction}\label{subsec:}
%-------------------------------------------------------------------------
%
%
\begin{equation}\label{eq:}
\div(\beta u) - \div(k \nabla u) + \psi(u) = f\quad\mbox{in $\Omega$},\qquad u=\udir \mbox{on $\GammaD$},\qquad k\dn{u}=\qd \mbox{on $\GammaN$}
\end{equation}
%
%
%~~~~~~~~~~~~~~~~~~~~~~~~~~~~~
\subsubsection{Courant element}
%~~~~~~~~~~~~~~~~~~~~~~~~~~~~~
%
%
\begin{align*}
\int_{\Omega}\psi(u)v +  \int_{\Omega}k \nabla u\cdot \nabla v -\int_{\Omega}u\beta\cdot\nabla v+\int_{\partial\Omega}\beta_n^+u v = \int_{\Omega}fv +\int_{\partial\Omega}|\beta_n^-|\udir v
\end{align*}
%
Piecewise constant approximation for $k$
%
\begin{align*}
\int_K \nabla \lambda_i \cdot \nabla \lambda_j = |K| \frac{n_i}{d_i}\cdot\frac{n_j}{d_j} = |K| \frac{|S_i|}{d|K|}\frac{|S_j|}{d|K|} n_i\cdot n_j
= \frac{1}{d^2|K|}\tilde n_i\cdot \tilde n_j\\
\\
-\int_K u \beta\cdot\nabla \lambda_i = u(x_K)|K| \beta\cdot\frac{n_i}{d_i} = u(x_K)\frac{1}{d}\beta\cdot\tilde n_i \\
- \int_S f \dn{v_i} = -f(x_S) |S| \nabla v_i \cdot n = f(x_S) |S| \frac{1}{d_i} n_i \cdot n= f(x_S) |S| \frac{|S_i|}{d|K|} n_i \cdot n
= f(x_S) \frac{\tilde n_i \cdot \tilde n}{d|K|}
\end{align*}
%

%
%-------------------------------------------------------------------------
\subsection{Turing}\label{subsec:}
%-------------------------------------------------------------------------
%
We consider a reaction-diffusion system on $\Omega=]-1,+1[^d$
%
\begin{equation}\label{eq:reacdiff}
\left\{\;
\begin{split}
\frac{\partial u}{\partial t}  -k_u\Delta u =& f(u,v)\quad\mbox{in $\Omega$},\qquad \frac{\partial u}{\partial n}=0\quad\mbox{on $\partial\Omega$},\\
\frac{\partial v}{\partial t}  -k_v \Delta v=&  g(u,v)\quad\mbox{in $\Omega$},\qquad \frac{\partial v}{\partial n}=0\quad\mbox{on $\partial\Omega$},
\end{split}
\right.
\end{equation}
%
Alan Turing discovered that the astonishing effect of destabilization by diffusion \cite{Turing52}, which leads to pattern formation 
\footnote{For an introduction and references see \url{https://en.wikipedia.org/wiki/Reaction–diffusion_system}.}
%
\begin{equation}\label{eq:}
f(u,v) = (a-u) - \psi(u,v),\qquad  g(u,v)=\psi(u,v).
\end{equation}
%
An equilibrium point satisfies
%
\begin{align*}
u^* = a,\qquad \psi(a,v^*)=0
\end{align*}
%
The linear stability analysis is based on the Jacobian
%
\begin{align*}
\begin{bmatrix}
-1 -\psi'_u & -\psi'_v\\ \psi'_u & \psi'_v
\end{bmatrix}
\quad\Rightarrow\quad \operatorname{tr} = \psi'_v-\psi'_u-1,\quad \det= -\psi'_v
\end{align*}

The brusselator is given by
%
\begin{equation}\label{eq:brusselator}
\psi(u,v) = bu - u^2v\qquad ( \psi'_u = b-2uv,\quad \psi'_v=-u^2).
\end{equation}

An equilibrium point of (\ref{eq:brusselator}) necessarily satisfies $u^*=a$ and $v^*=b/a$. We now have
%
\begin{align*}
\operatorname{tr} = -b + 2uv -u^2-1 ,\quad \det = u^2
\end{align*}
%
and for the equilibrium point
%
\begin{align*}
\operatorname{tr}^* = -1 - b +2b -a^2 = b-1-a^2,\quad {\det}^* = a^2
\end{align*}
%
We conclude that a  Hopf bifurcation appears if $b>a^2+1$.

%
%~~~~~~~~~~~~~~~~~~~~~~~~~~~~~
\subsubsection{The influence of diffusion}
%~~~~~~~~~~~~~~~~~~~~~~~~~~~~~
%
We consider an expansion into eigenfunctions of the Laplace operator withe eigenvalues $l\ge0$. The for the frequency $l$ we have the Jacopian
%
\begin{align*}
\begin{bmatrix}
-1 -\psi'_u - k_ul& -\psi'_v\\ \psi'_u & \psi'_v - k_vl
\end{bmatrix}
\quad\Rightarrow\quad \operatorname{tr} = \psi'_v-\psi'_u-1- (k_u+ k_v)l,\\ 
\det=  (1+k_ul)(k_vl-\psi'_v) +k_vl\psi'_u  = k_uk_v l^2 +(k_v(\psi'_u+1)-k_u\psi'_v)l - \psi'_v
\end{align*}
For the brusselator we have at the eqilubrium
%
\begin{align*}
\det= k_uk_v l^2 +(k_u a^2-k_v(b-1))l +a^2.
\end{align*}
%
The discriminant is
%
\begin{align*}
\Delta = (k_u a^2-k_v(b-1))^2 - 4 a^2 k_uk_v
\end{align*}
%
which is positive ($a=1$) for
%
\begin{equation}\label{eq:}
b \le b^* = 1 + \frac{k_u}{k_v} + 2 \frac{\sqrt{k_u}}{\sqrt{k_v}}
\end{equation}
%
The critical frequency is
%
\begin{align*}
l^*  = \frac{k_v(b^*-1)-k_u}{2k_uk_v} = 2 k_u^{-1/2}k_v{-3/2}
\end{align*}
%





The data for our test problems are
%
\begin{align*}
&\mbox{(cas 1)}\qquad a = 1.0\;,\quad b=2.1, \quad k_u = 0.0\;, \quad k_v = 0.0\;,\quad T=20.0\\
&\mbox{(cas 2)}\qquad a = 1.0\;,\quad b=1.9, \quad k_u = 0.0001\;, \quad k_v = 0.01\;,\quad T=20.0\\
&u_0(x) = \begin{cases}
1 & \mbox{if for all $i$ $x_i \in [-0.4,0.0]$}\\
0 & \mbox{else}
\end{cases}
\qquad
v_0(x) = \begin{cases}
1 & \mbox{if for all $i$ $x_i \in [-0.2,0.2]$}\\
0 & \mbox{else}
\end{cases}
\end{align*}
%


%==========================================
\printbibliography
%==========================================
\end{document}
%==========================================
